This bachelor thesis is about writing a fast and interactive 3D visualization library for Julia.
Special care was taken to make all libraries easy to use and to extend.
A crucial choice to achieve this was the choice of the programming language, which needs to be fast and high-level at the same time.
Julia is a novel programming language for scientific computing which promises to match C speed while offering a concise coding style. 
This makes Julia a great fit for implementing a scientific visualization library, as speed is crucial for smooth animated visualizations.
All parameters and also the data of the visualizations can be animated via Signals, which makes it easy to create interactive visualizations.
Simple \ac{GUI} elements were implemented as well, including text fields, sliders and color pickers. 
With these elements, Romeo was implemented. It is a simple scripting environment that allows you to execute Julia code and visualize and interact with all variables of the script.
All libraries are implemented in Julia and the \ac{OpenGL}. 
Like this state of the art speed was achieved opening the world to scientists who have to work with large datasets.
