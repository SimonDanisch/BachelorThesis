This bachelor thesis introduces Romeo, a prototype for an interactive scripting environment.
Romeo allows the user to execute Julia code and visualize and interact with all variables of the script.
Special care has been taken to make all components easy to use and extendable.
A crucial choice was the programming language, which needs to be both fast and high-level.
Julia is a novel programming language for scientific computing which promises to match the speed of C, while offering a concise coding style. 
This makes Julia a great fit for implementing a scientific visualization library, as speed is crucial for smoothly animated visualizations.
The visualization library, named GLVisualize, was implemented seperately from Romeo, to make it usuable for other similar projects. 
GLVisualize greatest accomplishment is to offer a simple way to animate all data via Signals while offering state of the art performance.
It also offers \ac{GUI} elements, including text fields, sliders and color pickers. 
Romeo utilizes the functionality of GLVisualize, allowing it to be a very small library.
All libraries are implemented in Julia and the \ac{OpenGL}.
This allows Romeo users to achieve top performance even for large datasets, while being able to stay in a high-level language for all tasks.
