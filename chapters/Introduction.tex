\section{Introduction}
% percent sign is for comments!
This Bachelor Thesis is about writing a fast and interactive 3D visualization environment for scientific computing.
The name of the library is Romeo and other libraries have been developed in order to implement Romeo. 
The focus is on usability, applied to all the different interfaces, ranging from abstract \ac{API} interfaces to graphical user interfaces. 
Julia was chosen for implementing Romeo, as it offers a unique combination of speed, high-level programming style and a lot of functionality for scientific computing.
The ultimate goal is to make scientific computing more accessible to the user.
\ac{GUI} elements and editable text fields are supplied, which can be used for interaction with the data and executing scripts.
The system offers a default visualization for every data type. This together with the widgets forms a basis for visual debugging.

The introduction is structured in the following way:
an introduction to the general field of research and its challenges and problems relevant to this thesis will be extracted.
Finally, this chapter will conclude with a solution to the problem, how to measure the success and give an outlook on the structure of the entire Bachelor Thesis.


\subsection{Scientific Computing}

Scientific computing is the study of computing that evolves around all kind of scientific research.
It is a very broad field involving a lot of different challenges. 
In some areas such as particle physics, the problems are computationally demanding and can only be solved with the help of supercomputers.
In other areas like robotics, it is important to be efficient because the algorithms are running on embedded systems with limited resources. 
In a lot of other areas, speed is not that important, but it may be that the algorithm in itself is very difficult to comprehend. 
So the more comprehensible an algorithm can be written in a programming language, the easier it will be to implement the algorithm without errors.
In any case, programming itself is secondary to the researcher.
It can be expected that a researcher only has rudimentary programming skills. 
Even if they area professional programmers, it is often inefficient to implement complex algorithmic programming problems.

So things like manual memory management and difficult design patterns with a lot of unnecessary code should be avoided in scientific computing.
This has led to the rise of programming languages and tools specifically tailored to scientific computing.
The most prominent examples include Mathematica, R, and Matlab. 
Python could be in this list as well, but the scientific computing part is only realized by third party libraries while Python itself is a multi-purpose language.
The others all aim to provide a simple syntax for linear algebra and statistical code while taking away difficult tasks like memory management. 
Also, they come with a rich standard library, which means most research can be done without loading any additional module, which makes them great tools for rapid prototyping.
At the current state, the speed of these languages suffers from poor performance with the high level of abstraction. 
As a consequence, a lot of the performance critical core is written in another language like C/C++ or Fortran. 
This leads us straight to the field of research and the problems that are addressed in this thesis.


\subsection{Field of Research and Problem}


In a slow high-level programming language like Matlab, one needs to switch to a fast multi-purpose language, as soon as one needs to do something with high-performance demands.
In this case, one is losing all the advantages of the high-level scientific computing language.
A pattern which has evolved out of this dilemma is to prototype in a nice high-level language and as soon as the algorithm has been confirmed to work, it gets rewritten in a fast language.

This introduces great development costs and makes it harder to further develop the algorithm.

One of the first languages promising to solve this dilemma for scientific computing is Julia. 
It is designed to be high-level and optimized for the work of scientific computing while approaching the speed of statically compiled languages like C.

This thesis is about bringing performance and usability together in the realms of scientific computing and 3D visualizations.
These two demands are opposing concepts. 
One is about bringing tasks into a form which is best understandable to humans, and the other is about transforming a task to make it perform well on a specific computer architecture.
These two tasks could not be more different. 
For humans, data and algorithms become understandable if they are high-level and represented visually, auditorial or tactile together with immediate feedback. 
It is the task of making problems accessible to a human, who has evolved his capabilities in order to survive and find food and not to create complex algorithms.
Computers, on the other hand, love to have their registers filled optimally, move memory to smaller and faster caches and dislike random access to memory. That is all they care about, whether a human understands this or not.

To close this gap, compilers have been created. They are translators between human understandable languages to machine instructions.
This is just the first step and many more are needed to create an enjoyable user experience.
These steps range from introducing graphical user interfaces, novel input devices like the mouse, understandable visualizations and so forth.
All these advances have made computers usable for people who do not have an education in computer science.
In this thesis, the field is scientific computing, which still has quite a lot of barriers for novel users.
Scientific computing is usually about implementing mathematical equations, complex algorithms and manipulating and analyzing data.
Most research is done in some specialized, high-level scientific computing language.
Besides the previously discussed performance problems, the lack of easily usable, extendable and fast visualization libraries also poses a problem.
Most state of the art visualization libraries use C++ at their performance critical core, they are closed source or they are simple toy libraries, which can not be used for projects with higher demands.

Using C++ introduces complexity and performance bottlenecks when interfacing with other languages. This is especially true for languages that are not binary compatible to C++. 
The next problem occurs, when the library does not offer the needed functionality and the programmer has to step in and extend the library. If it is closed source or one has to switch the language for that, this will either be not possible or introduce additional work. 
It is nice to have a library in which one can get results very quickly. 
But it is very frustrating when one needs to switch to another library for more serious projects, as previously learned concepts and work has to be discarded.
So it is desirable to have a library which scales well from small projects to big projects.

Finally, one often does not have GUI elements. 
Even if there are GUI elements, they might come from a different package (possibly written in yet another language) or they are complicated to use.
All in all, this makes it hard for researchers to visualize and interact with their data and creating solutions which are tailored to the research problem.

\vspace{1em}
\begin{minipage}{\linewidth}
    \centering
    \includegraphics[width=0.7\linewidth]{graphics/surfaces.png}
    \captionof{figure}[Volume Visualization]{different visualizations of $f(x,y,z)=\sin(\frac{x}{15})+\sin(\frac{y}{15})+\sin(\frac{z}{15})$, visualized with Romeo. From left to right: Isosurface with isovalue=0.76, Isosurface with isovalue=0.37, maximum value projection}
    \label{fig:volume}
\end{minipage}
\vspace{1em}

3D visualization libraries play an integral role in scientific computing.
The computer is much better than us at executing algorithms, displaying them and helping us to understand them.
A good visualization can be the primer to understanding problems better.
Consider the following function: $f(x,y,z)=\sin(\frac{x}{15})+\sin(\frac{y}{15})+\sin(\frac{z}{15})$. 
It describes a 3D volume mathematically. 
This is a simple function, which is already not that easy to interpret. In Figure \ref{fig:volume} one can see different visualizations of \textit{f}.
If you can interact with this visualization by moving through the iso-values or coloring certain areas, which will make the function more understandable.
This deeper understanding is crucial for identifying problems in the underlying math, extending the function, or explaining it to other people. 
Making problems more understandable further opens the gates of scientific computing to novel users.

Also, a lot of debugging problems can be a lot easier to solve with a good visualization. 
If one writes an algorithm that calculate normals of faces on a mesh, the errors in the math becomes obvious when visualizing the normals.
Performance bottlenecks in a call-graph can be seen easily if the graph is color-coded for the execution times of a call. 
Making these tasks enjoyable can help to get an easy start in scientific computing.

In summary, the software in this thesis focuses on research which involves writing short scripts while playing around with some parameters and visualizing the results via built-in, or user defined, visualization routines.
An example would be a materials researcher, who is investigating different 3D shapes and materials and their reaction to pressure.
The researcher would need to read in the 3D object they want to analyze, have an easy way to tweak the material parameters and it would be preferable to get instant feedback on how the pressure waves propagate through the object. Also while doing all this, they may want to modify the script that calculates the pressure.
There are quite a few libraries out there offering this, but In the next chapter, the contributions will be listed, which make Romeo ideal for this task.


\subsection{Contribution}

The main contribution of this thesis is writing Romeo in Julia, which offers the following advantages.

Julia is a high-level language and effort has been put into creating a concise architecture. From this follows, that the development cycles can be very short and the library is easy to extend.

Romeo targets researchers that want to write everything in one language.
As Julia is fast and the library is also written in Julia, this will enable researchers to stay in the same language for their project. 
This makes it easy to create visualization pipelines in which every routine is as fast as it can get. 
On top of that Romeo uses modern OpenGL, which allows to achieve fast, hardware accelerated 3D renderings. [do I need a reference to the OpenGL chapter here?!]
Also, the researcher can extend the library in the same language they are already working in. 
In the case of Julia and Romeo, this is especially easy, as every project involved is open source and directly accessible.
This allows for the flexibility and transparency which is needed for big projects.

On top of that, the library makes it simple to interact with complex algorithms via widgets and forms a basis for visual debugging. 
This comes with an ease of use, which would be hard to achieve if the library was not that deeply embedded in Julia.


\subsection{Outlook}
%Structure of BA and a few worts on the results 
This thesis will continue with the chapter \textbf{Background}, which gives the reader an overview over Julia and similar languages with their respective 3D visualization libraries.
After this, the requirements for Romeo will be laid out and how to measure if the requirements were met.
Then the technologies used will be explained, which builds the basis for discussing the implementation.
In the chapter \textbf{Results and Discussion}, one will find if the requirements were met and how Romeo compares to similar software.
This chapter leads straight to the conclusion which will summarize this thesis.
[is this enough!?]