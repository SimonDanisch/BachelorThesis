\subsection{Performance Analysis}
benchmarks, benchmarks, benchmarks
\subsection{Extendability Analysis}
The modular design of Romeo has proven to be very effective and the goal of reusability has already proven itself.
Most of the created modules are already used independantly by different people.
GLVisualize is used by myself for two packages, namely GLPLot, a scientific plotting package for Julia and a for a prototype of a file explorer. 
It got forked by several users to create their own dynamic visualization packages.
The same applies for ModernGL and GLAbstraction. Most other used packages are at least used by one other project.
This indicates, that the abstraction and modularity is well designed, so that all the modules can function on their own.

The only exception is GLWindow, which has been used just indirectly through my other packages. 
This can mean three things.
First, it is badly abstracted and doesn't cleanly wrap one use case.
Secondly, it can be, that the use case is not entirely clear to other people, which would not be a big surprice considering the minimal amount of documentation for GLWindow.
And finally, considering the small group of people deveoping graphics for Julia, it could be that they simply don't need the lower level functionality of GLWindow and instead rely on my other packages that use GLWindow.

This kind of modularity guarantees a broad user and developper base, which in turn results in rich functionality and stability.
From further analyzing the github repository written by me, one can find out that there is a lack of documentation.
This hinders people from contributing and using the packages, but could not been 

\subsubsection{IJulia}
So IJUlia is written in:

using ZMQ(C++), D3, 
Three.js
JavaScript 62.4	 HTML 26.4	 Python 6.9,	 C++ 1.9	 C 1.3,	 GLSL 0.6
D3
JavaScript 95.6	 CSS 4.3
IPython
Python 78.5	 JavaScript 15.1	 HTML 5.0	 Other 1.4

\subsubsection{Paraview and VTK}


\begin{table}[h]
	\centering
	\caption{Paraview, language statistic}
    \sffamily 
	\begin{tabular}{ l|l|l|l|l }
		\hline
		\textbf{Language}              &\textbf{files} &\textbf{blank} &\textbf{comments} & \textbf{code} \\
		\hline
		C++                            &2037          &70003         &86594         & 391121\\
		C/C++ Header                   &1937          &48345         & 93434        & 141581\\
		C                              & 273          &35843         & 17101        & 135937\\
		XML                            & 275          & 1930         &  3521        &  59030\\
		Fortran 77                     &  67          &   28         & 18039        &  39116\\
		Python                         & 209          & 5883         &  8719        &  21935\\
		CMake                          & 443          & 3705         &  6185        &  20025\\
		Javascript                     &  20          & 1285         &  1847        &   7982\\
		CSS                            &  23          &  750         &   251        &   4827\\
		HTML                           &  26          &  240         &  1692        &   2328\\
		JSON                           &  13          &    2         &     0        &   2162\\
		yacc                           &   1          &  207         &   138        &    881\\
		Bourne Again Shell             &  19          &  186         &   347        &    799\\
		make                           &   8          &  248         &    90        &    734\\
		Bourne Shell                   &  18          &  158         &   116        &    708\\
		XSLT                           &   3          &   46         &    17        &    388\\
		CUDA                           &   1          &   58         &   184        &    318\\
		Pascal                         &   2          &   69         &   102        &    228\\
		\hline
		SUM:                           &5375         &168986         &238377         &830100\\
		\hline
	\end{tabular}
    \normalfont
    \label{table:ParaviewStatistic}
\end{table}

\begin{table}[h]
	\centering
	\caption{VTK, language statistic}
    \sffamily 
	\begin{tabular}{ l|l|l|l|l }
		\hline
		\textbf{Language}&      \textbf{files}&\textbf{blank}& \textbf{comment}& \textbf{code}\\
		\hline
		C++                                   &3845         &203851         &179827        &1278279\\
		C                                     &1103         &130996         &289623        &707122\\
		C/C++ Header                          &3489         &103162         &246368        & 382728\\
		Python                                &1681         & 88983         &121122        & 258787\\
		Tcl/Tk                                & 573         & 11052         &  7830        &  48213\\
		CMake                                 & 739         &  4715         &  7424        &  35956\\
		Javascript                            &  47         &  6941         &  6747        &  33098\\
		CSS                                   &  33         &  1476         &   323        &  18100\\
		XML                                   &  10         &    17         &    36        &   8337\\
		Objective C++                         &  20         &  1210         &  1372        &   5601\\
		m4                                    &   3         &   660         &    83        &   4922\\
		yacc                                  &   3         &   726         &   570        &   4852\\
		HTML                                  &  25         &   553         &   531        &   4313\\
		Java                                  &  50         &   912         &  1192        &   4239\\
		Cython                                &  20         &   848         &  1625        &   3484\\
		Perl                                  &  11         &   939         &   950        &   3119\\
		JSON                                  &   3         &     5         &     0        &   2658\\
		Windows Resource File                 &  21         &   333         &   380        &   1835\\
		lex                                   &   3         &   215         &   162        &   1510\\
		DTD                                   &   3         &   435         &   477        &   1335\\
		Assembly                              &  13         &   202         &     0        &    936\\
		Bourne Again Shell                    &  16         &   191         &   333        &    866\\
		CUDA                                  &   6         &   113         &    77        &    740\\
		Bourne Shell                          &  15         &    64         &   122        &    380\\
		make                                  &   5         &    54         &   187        &    170\\
		IDL                                   &   1         &     0         &     0        &    150\\
		Windows Module Definition             &   3         &     3         &     0        &    142\\
		JavaServer Faces                      &   3         &    26         &     0        &     88\\
		Objective C                           &   2         &    13         &    18        &     17\\
		\hline
		SUM:                                 &11749         &558698         &867379        &2812005\\
		\hline
	\end{tabular}
    \normalfont
    \label{table:VTKStatistic}
\end{table}

\subsection{Usability Analysis}



\subsection{Discussion}
Have I reached my goals?